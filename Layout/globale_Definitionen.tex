% Meta-Informationen -----------------------------------------------------------
%   Definition von globalen Parametern, die im gesamten Dokument verwendet
%   werden k�nnen (z.B auf dem Deckblatt etc.).
%
%   ACHTUNG: Wenn die Texte Umlaute oder ein Esszet enthalten, muss der folgende
%            Befehl bereits an dieser Stelle aktiviert werden:
%            \usepackage[latin1]{inputenc}
% ------------------------------------------------------------------------------
\newcommand{\titel}{Umprogrammierung von EPROM-Modulen f�r den {\FLTester}}
\newcommand{\untertitel}{}%{TODO: und hier kommt der Untertitel}
\newcommand{\Bezeichnung}{Module f�r {\FLTester} programmieren}
\newcommand{\BezeichnungLang}{EPROM-Module f�r {\MarelliTester} �ndern}
%\newcommand{\Version}{V0.1}
\newcommand{\Dokumentart}{A N L E I T U N G}
\newcommand{\autor}{schlizb�da}
\newcommand{\jahr}{2016}

% verwendete Hardware
\newcommand{\FLTester}{F/L-Tester}
\newcommand{\MarelliTester}{Fiat Lancia Tester (Magneti Marelli)} % TODO: Internetbilder "FIAT" "LANCIA" "Tester" "MagnetiMarelli" 
\newcommand{\Batronix}{Batronix BARLINO II} % Programmierger�t
% verwendete Software
\newcommand{\ProgExpress}{Prog-Express} % Batronix Programmiersoftware

%%Steuerelemente von Software:
%\newcommand{\prompt}[1]{\Code{\textit{#1}}}
%\newcommand{\filenam}[1]{\Code{#1}}
%
%\newcommand{\button}[1]{\Code{[{#1}]}}
%\newcommand{\menuitem}[1]{\textbf{\textit{"{#1}"}}}
%\newcommand{\checkbox}[1]{\textbf{\textit{"{#1}"}}}

%Smileys:
\newcommand{\smiley}[1]{\includegraphics[width=0.3cm]{Bilder/smileys/{#1}}}
